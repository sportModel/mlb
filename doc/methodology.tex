\documentclass[a4paper,twoside,12pt]{article}

\usepackage{amsmath}
\usepackage{amssymb}
\usepackage{fullpage}
\usepackage{graphicx}
\usepackage{afterpage}
\usepackage{psfrag}
\usepackage[bf,small]{caption}
%\usepackage[symbol]{footmisc}

\title{Baseball Analysis: A Value-Created Approach}
\author{Patrick Breheny}
\date{}

\begin{document}  

\newcommand{\minisection}[1]{\noindent \emph{#1}}

\setlength{\voffset}{0.5in}

\maketitle

Any well-organized project needs an organizing principle, and the organizing principle of my approach to baseball analysis is that a perfectly average team of perfectly average players that pitches, hits, and fields in an average way will (on average) score an average number of runs, give up an average number of runs, and win half of their games.  However, any deviation from this average will change these expectations, and it is in this sense that players and teams have value -- i.e., some teams are better than others because you can expect them to win more games, some offenses are better than others because you can expect them to score more runs, etc.

I introduce the term \emph{value created} to refer to the steady accumulation of deviations from expectation.  As we will see, the idea of value creation is a powerful tool that allows us not only to compare teams but also to decompose the value that a team has created into the contributions of individual players and thus to obtain an objective measure of their value (at least, their value in terms of how they are currently utilized by their team).

Section 1 lays out the fundamental relationship between runs and wins that forms the basis for assigning the contributions of individual players to a team's wins and losses.  Section 2 changes the goal from that of modeling wins to that of modeling runs and discussing a modeling approach both for runs scored and runs allowed.  The remaining sections deal with decomposing the models presented in section 2 into models applicable to individual players.  Section 3 discusses the relatively easy task of assigning individual value creation for scoring runs; section 4 tackles the much tougher question of assigning individual value creation for allowing runs.

\section{Runs and Wins}

The objective of baseball is to win games, so the final measure of any player or team must be their ability to do so.  However, associating individual actions with wins directly is a difficult chore, so our approach will consist of associating individual actions with runs.  Of course, this apporach is only valid if runs and wins are tightly correlated.  This section investigates the relationship between runs and wins.

Let $dW$ denote the change in the number of wins with respect to a perfectly average team (i.e. a team that finishes 81-81 has $dW = 0$, an 82-80 team has $dW = 1$, an 80-82 has $dW = -1$).  Based on our organizing principle, a team that scores and allows an average number of runs should expect $dW = 0$.  Thus, we can assess the value of scoring and allowing runs with the following model:
\begin{equation*}
  dW = \beta_{R} dR
\end{equation*}
where $dR = dRS - dRA$, with $dRS$ and $dRA$ referring, respectively to deviations from the expected number of runs scored and allowed.

A natural question, and one we shall return to often in this document, is: what should be expected?  In this case, how many runs should we expect a team to score?  In general, this question is open to debate, and how we answer it will have definite effects on the conclusions we draw.  I will present the calculations that go into the website, but they are certainly not the only approach that falls into this value-created framework.  If my assumptions strike you as false, keep that in mind when you view the conclusions that come from them.

Here, my approach is that the number of runs we expect a team to score is $eRS = G \cdot \overline{RSG}$, where $\overline{RSG}$ is the average number of runs per game scored in the major leagues that year and $G$ is the number of games that team has played.  This is certainly a reasonable approach, although if we were analyzing data from the first few weeks of a major league season, we might wish instead to use averages from the last year, as they will be more stable.

Thus, our starting point is
\begin{align}
  dW &= \beta_{R} dR
 \intertext{where}
  dR &= dRS - dRA \\
  dRS &= RS - eRS \\
  dRA &= RA - eRA \\
  eRS &= G \cdot \overline{RSG} \\
  eRA &= G \cdot \overline{RAG}
\end{align}

For this and future models, I use team-level data for all MLB teams from 1951-2006 to estimate regression coefficients using ordinary least squares fitting.  The fitted value of $\beta_{R}$ is .100.  The $r^2$ of this model is 90\%; i.e. $dR$ explains 90\% of the variability in $dW$, with the remaining 10\% attributable to luck in this model (clutch play in the eyes of others).

The advantage of building a model in this way (as opposed, say, to that proposed by Palmer \cite{palmer}) is that our conclusions are naturally \emph{scale-invariant} -- the above model has no intercept, and nothing ties it to an arbitrary 162-game season; the model is just as valid at evaluating an entire season as it is a single game.  For instance, if a team outscores its opponents by 10 runs over the course of 10 games, they can expect to go 6-4; if they outscore their opponents by 10 runs over 100 games, they can expect to go 51-49.  This seems quite reasonable.

Or, to look at it another way, if an individual's actions resulted in an extra run scored (or prevented), that action is worth a tenth of a win.  As noted earlier, it is difficult to link a player's actions to wins and losses directly, but it is much easier to link those actions to runs.  Equation (1) provides the means for translating the results that follow into player wins.

\section{Modeling Runs}

In this section, we continue to consider baseball data at the team level, but in addition to our model for wins, we will build two additional models: one for runs scored and one for runs allowed.  We will continue to work in the scale-invariant framework introduced in Section 1.

My approach to modeling here is fairly straightforward: any team-level statistic gathered for all teams from 1951-2006 that intuitively would seem to have an effect on runs scored or allowed was included in the model, regardless of whether its inclusion is significant from a statistical sense.  With $n=$ observations, parsimony was not an overriding concern.  However, these statistics were chosen with an eye towards the ability to later separate the contributions to these statistics by individual players.  For example, it makes no sense to include ERA as a statistic, since that will completely dominate the prediction of dRA and cannot be separated into pitching and fielding components.  Credit for strikeouts, on the other hand, can be given entirely to the pitcher.  These concepts will be explained in greater detail in section 4.

\begin{align}
  dRS &= \beta_{1B} d1B + \beta_{2B} d2B + \beta_{3B} d3B + \beta_{HR} dHR\\
      &+ \beta_{BB} dBB + \beta_{SB} dSB + \beta_{CS} dCS \notag \\
  dRA &= \beta_{SOA} dSOA + \beta_{BBA} dBBA + \beta_{HRA} dHRA + \beta_{BIPH} dBIPH\\
      &+ \beta_{SBA} dSBA + \beta_{CSA} dCSA \notag %\\
%      &+ \sum_{i=1}^9 \beta_{E_i} dE_i + \sum_{i=7}^9 \beta_{A_i} dA_i \notag
\end{align}

where I follow the common convention of appling an ``A'' for ``allowed'' at the end of a defensive statistic to distinguish it from its corresponding offensive statistic.  BIPH stands for hits on balls in play, an ``in play'' ball being defined as one that is not a strikeout, walk, or home run [reference McCracken].  The collection ${E_i}$ refers to errors made at each of the nine positions, and the collection ${A_i}$ refers to outfield assists.  Least squares estimators of all the coeficients in the above model are presented in Table 1.

Table 1 here

The $r^2$ of each of the above models is 90\%.  Thus, the factors above explain 90\% of a team's ability to score and prevent runs.  It should be noted here that other linear regression-based systems report a $r^2$ higher than 90\% -- however, that should not be taken as indicative of their superiority!  Other models are based on predicting runs, not dRS.  Runs are easier to predict -- number of games played explains by itself 54\% of the variability in runs scored.  On the other hand, it is impossible to say based on volume of games alone whether a team will score above or below the average number of runs in those games.  Furthermore, for modeling defensive contributions, the real issue is not the ability to predict runs (refer to the ERA comment above), but rather the ability to separate defensive contributions from pitchers and fielders.  The simple factors above explain the vast majority of differences in runs scored and allowed; one could probably construct a much more complicated model and explain an extra 1\% of the variability, but I prefer the simplicity of the above model.

In some respects, these numbers are rather similar to those found by Palmer.  More comments.  No outs.

\section{Individual Value Creation from Runs Scored}



\section{Individual Value Creation from Runs Allowed}


\section{Fielding and Pitching}


\begin{thebibliography}{9}
\bibitem{palmer} Thorn and Palmer
\end{thebibliography}

\subsection*{Appendix: Data}
Before we start proposing models, a choice must be made regarding the set of data we will fit our model to (e.g., all MLB data throughout history?  Or from 1945-present?  1995-present?).  There are tradeoffs here: fit the model to recent data only, and we lose the ability to apply the model to the study of past seasons.  In addition, we lower our sample size, increasing the variability of the model fit.  On the other hand, if we fit the model to all historical data, we risk applying antiquated baseball standards to today's hitters (an average hitter from the 1910s is very different than an average hitter today).

To some extent, this choice reflects our interests.  I will assume that the predominant interest of this methodology is to study current baseball players, and will therefore always be willing to sacrifice historical interests for the accuracy of modern-day conclusions.  However, there is nothing stopping a person from applying the identical methodology and simply using a different data set; indeed, this is a trivial parameter to change, although it will affect all the parameters (albeit usually only slightly) derived here.

For the rest of this document, I will take our model data to be the years 19??-2006 -- see figure 1 for justification.

\end{document}
